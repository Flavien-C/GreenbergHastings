\documentclass[12pt, a4paper]{article}


\usepackage[top=1.5cm, left=1.5cm, bottom=1.5cm, right=1.5cm]{geometry}
\usepackage{graphicx}

\usepackage[utf8]{inputenc}
\usepackage[T1]{fontenc}
\usepackage[francais]{babel}


\begin{document}
    
    \begin{titlepage}
        \begin{minipage}{.4\linewidth}
            \includegraphics{img/logo.png}
        \end{minipage}
        \hfill
        \begin{minipage}{.4\linewidth}
            \begin{flushright}
                2017-2018
            \end{flushright}
        \end{minipage}
        
        \vspace{7cm}
        
        \begin{center}
            \textbf{\Huge{Rapport d'expérimentation sur la dynamique des automates cellulaires}}
        \end{center}
        
        \vspace{10cm}
        
        \begin{minipage}{.4\linewidth}
            CASTEL Flavien 21405613\\
            VOISIN Dan 21502515
        \end{minipage}
        \hfill
        \begin{minipage}{.4\linewidth}
            \begin{flushright}
                L2 Informatique
            \end{flushright}
        \end{minipage}
    
    \end{titlepage}
    
    
    \section{Premières évolutions de l'automate de Greenberg-Hastings}
        Sur le premier modèle de Greenberg-Hasting avec un automate cellulaire à trois états et d'une taille 80$\times$80 cases, l'évolution de cet automate ressemble aux figures suivantes:
        \begin{center}
            \begin{minipage}{.17\linewidth}
                \includegraphics[scale=0.15]{img/part1/step1.png}
            \end{minipage}
            $\rightarrow$
            \begin{minipage}{.17\linewidth}
                \includegraphics[scale=0.15]{img/part1/step2.png}
            \end{minipage}
            $\rightarrow$
            \begin{minipage}{.17\linewidth}
                \includegraphics[scale=0.15]{img/part1/step3.png}
            \end{minipage}
            $\rightarrow$
            \begin{minipage}{.17\linewidth}
                \includegraphics[scale=0.15]{img/part1/step4.png}
            \end{minipage}
            $\rightarrow$
            \begin{minipage}{.17\linewidth}
                \includegraphics[scale=0.15]{img/part1/step5.png}
            \end{minipage}
        \end{center}
    
    
    \section{Phénomènes d'émergences sur un automate donné}
        On utilise maintenant un modèle généraliste de Greenberg-Hastings. On utilise comme paramètres huit états possibles par cellules, un rayon de 3 cases, un seuil de 5 et une taille de 80$\times$80 cases.
        \begin{center}
            \begin{minipage}{.17\linewidth}
                \includegraphics[scale=0.15]{img/part2/step1.png}
            \end{minipage}
            $\rightarrow$
            \begin{minipage}{.17\linewidth}
                \includegraphics[scale=0.15]{img/part2/step2.png}
            \end{minipage}
            $\rightarrow$
            \begin{minipage}{.17\linewidth}
                \includegraphics[scale=0.15]{img/part2/step3.png}
            \end{minipage}
            $\rightarrow$
            \begin{minipage}{.17\linewidth}
                \includegraphics[scale=0.15]{img/part2/step4.png}
            \end{minipage}
            $\rightarrow$
            \begin{minipage}{.17\linewidth}
                \includegraphics[scale=0.15]{img/part2/step5.png}
            \end{minipage}
            $\rightarrow$
            \begin{minipage}{.17\linewidth}
                \includegraphics[scale=0.15]{img/part2/step6.png}
            \end{minipage}
            $\rightarrow$
            \begin{minipage}{.17\linewidth}
                \includegraphics[scale=0.15]{img/part2/step7.png}
            \end{minipage}
            $\rightarrow$
            \begin{minipage}{.17\linewidth}
                \includegraphics[scale=0.15]{img/part2/step8.png}
            \end{minipage}
            ...
        \end{center}
        On remarque que l'automate subit d'abord une phase d'évolution chaotique, puis se stabilise pour répéter le cycle d'évolution de la seconde ligne à l'infini.
    
    
    \section{Proposition d'automates créant des phénomènes d'émergences}
        \subsection{Vers l'évolution de plusieurs cellules}
            Cet automate a une taille de 500$\times$500 cases, huit états possibles, un rayon de quatre cases, et un seuil de huit.
            \begin{center}
                \begin{minipage}{.17\linewidth}
                    \includegraphics[scale=0.15]{img/part3/1/step1.png}
                \end{minipage}
                $\rightarrow$
                \begin{minipage}{.17\linewidth}
                    \includegraphics[scale=0.15]{img/part3/1/step2.png}
                \end{minipage}
                $\rightarrow$
                \begin{minipage}{.17\linewidth}
                    \includegraphics[scale=0.15]{img/part3/1/step3.png}
                \end{minipage}
                $\rightarrow$
                \begin{minipage}{.17\linewidth}
                    \includegraphics[scale=0.15]{img/part3/1/step4.png}
                \end{minipage}
            \end{center}
            Comme dans l'automate de la partie précédente, l'automate se stabilise pour répéter ensuite le même cycle d'évolution.
        
        
        \subsection{Vers la mort de l'automate}
            Cet automate a une taille de 80$\times$80 cases, dix états possibles, un rayon de quatre cases, et un seuil de huit.
            \begin{center}
                \begin{minipage}{.17\linewidth}
                    \includegraphics[scale=0.15]{img/part3/2/step1.png}
                \end{minipage}
                $\rightarrow$
                \begin{minipage}{.17\linewidth}
                    \includegraphics[scale=0.15]{img/part3/2/step2.png}
                \end{minipage}
                $\rightarrow$
                \begin{minipage}{.17\linewidth}
                    \includegraphics[scale=0.15]{img/part3/2/step3.png}
                \end{minipage}
                $\rightarrow$
                \begin{minipage}{.17\linewidth}
                    \includegraphics[scale=0.15]{img/part3/2/step4.png}
                \end{minipage}
            \end{center}
            Cet automate va se stabiliser en ayant toutes ces cellules dans le même état. De ce fait, l'automate ne va plus évoluer.
\end{document}